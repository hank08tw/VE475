\documentclass{article}

\usepackage{fancyhdr}
\usepackage{extramarks}
\usepackage{amsmath}
\usepackage{amsthm}
\usepackage{amsfonts}
\usepackage{tikz}
\usepackage[plain]{algorithm}
\usepackage{algpseudocode}

\usetikzlibrary{automata,positioning}

%
% Basic Document Settings
%

\topmargin=-0.45in
\evensidemargin=0in
\oddsidemargin=0in
\textwidth=6.5in
\textheight=9.0in
\headsep=0.25in

\linespread{1.1}

\pagestyle{fancy}
\lhead{\hmwkAuthorName}
\chead{\hmwkClass\ (\hmwkClassInstructor\ \hmwkClassTime): \hmwkTitle}
\rhead{\firstxmark}
\lfoot{\lastxmark}
\cfoot{\thepage}

\renewcommand\headrulewidth{0.4pt}
\renewcommand\footrulewidth{0.4pt}

\setlength\parindent{0pt}

%
% Create Problem Sections
%

\newcommand{\enterProblemHeader}[1]{
    \nobreak\extramarks{}{Problem \arabic{#1} continued on next page\ldots}\nobreak{}
    \nobreak\extramarks{Problem \arabic{#1} (continued)}{Problem \arabic{#1} continued on next page\ldots}\nobreak{}
}

\newcommand{\exitProblemHeader}[1]{
    \nobreak\extramarks{Problem \arabic{#1} (continued)}{Problem \arabic{#1} continued on next page\ldots}\nobreak{}
    \stepcounter{#1}
    \nobreak\extramarks{Problem \arabic{#1}}{}\nobreak{}
}

\setcounter{secnumdepth}{0}
\newcounter{partCounter}
\newcounter{homeworkProblemCounter}
\setcounter{homeworkProblemCounter}{1}
\nobreak\extramarks{Problem \arabic{homeworkProblemCounter}}{}\nobreak{}

\newenvironment{homeworkProblem}{
    \section{Problem \arabic{homeworkProblemCounter}}
    \setcounter{partCounter}{1}
    \enterProblemHeader{homeworkProblemCounter}
}{
    \exitProblemHeader{homeworkProblemCounter}
}

%
% Homework Details
%   - Title
%   - Due date
%   - Class
%   - Section/Time
%   - Instructor
%   - Author
%

\newcommand{\hmwkTitle}{Homework\ \#2}
\newcommand{\hmwkDueDate}{May 31, 2019}
\newcommand{\hmwkClass}{Introduction to Cryptography}
\newcommand{\hmwkClassTime}{}
\newcommand{\hmwkClassInstructor}{Professor Manuel}
\newcommand{\hmwkAuthorName}{ShiHan Chan}

%
% Title Page
%

\title{
    \vspace{2in}
    \textmd{\textbf{\hmwkClass:\ \hmwkTitle}}\\
    \normalsize\vspace{0.1in}\small{Due\ on\ \hmwkDueDate\ at 3:10pm}\\
    \vspace{0.1in}\large{\textit{\hmwkClassInstructor\ \hmwkClassTime}}
    \vspace{3in}
}

\author{\textbf{\hmwkAuthorName}}
\date{}

\renewcommand{\part}[1]{\textbf{\large Part \Alph{partCounter}}\stepcounter{partCounter}\\}

%
% Various Helper Commands
%

% Useful for algorithms
\newcommand{\alg}[1]{\textsc{\bfseries \footnotesize #1}}

% For derivatives
\newcommand{\deriv}[1]{\frac{\mathrm{d}}{\mathrm{d}x} (#1)}

% For partial derivatives
\newcommand{\pderiv}[2]{\frac{\partial}{\partial #1} (#2)}

% Integral dx
\newcommand{\dx}{\mathrm{d}x}

% Alias for the Solution section header
\newcommand{\solution}{\textbf{\large Solution}}

% Probability commands: Expectation, Variance, Covariance, Bias
\newcommand{\E}{\mathrm{E}}
\newcommand{\Var}{\mathrm{Var}}
\newcommand{\Cov}{\mathrm{Cov}}
\newcommand{\Bias}{\mathrm{Bias}}

\begin{document}

\maketitle

\pagebreak

\begin{homeworkProblem}

    \textbf{Part One}
	Since gcd(17,101)=1, 17 and 101 are coprime integars. There exist two integars s,t satisfy the equation: $17*s+101*t=1$, and s is the inverse of 17 module 101. By applying extended euclidean algorithm, we can find s and t.\\
	101=17*5+16\\
	17=16*1+1\\
	16=1*16
	
	1=17-16*1\\
	1=17-(101-17*5)*1\\
	1=101*-1+17*6\\
	s=6,t=-1\\
	so the inverse of 17 module 101 is 6.
	
   	
    \textbf{Part Two}
	
	$12x \equiv  28$  mod $236$\\
	$12x - 28= 236n$ (n is a integar)\\
	$3x - 59n = 7$\\
	Since gcd(3,59)=1, 3 and 59 are coprime, there exists a integar solution pair (s,t) to the equation 3s+59t=1.\\
	Applying extended euclidean algorithm:\\
    59=3*19+2\\
    3=2*1+1\\
    2=1*2\\
    
    1=3-2*1\\
    1=3-(59-3*19)*1
    1=3*20+59*(-1)
    
    
    we can get s=20, t=-1.\\
    3*20+59*(-1)=1\\
    By multiplying 7 on both side, we can get:\\
    3*(140)+59*(-7)=7\\
    We can get one solution x=140.\\
    lcs(3,59)=3*59=177\\
    x=140+59*n=22+59*n' ($n,n' \in Z$)
    
    
    \textbf{Part Three}\\
   	Since $m\in [0,30], c\in [0,30]$, we can build a relationship table between plaintext m and ciphertext c.
   	\begin{tabular}{|c|c|c|c|c|c|c|c|}% 通过添加 | 来表示是否需要绘制竖线
   		\hline  % 在表格最上方绘制横线
   		m&c&m&c&m&c&m&c\\
   		\hline  %在第一行和第二行之间绘制横线
   		0&0&1&1&2&4&3&17\\
   		\hline % 在表格最下方绘制横线
   		4&16&5&5&6&6&7&28\\
   		\hline % 在表格最下方绘制横线
   		8&2&9&10&10&20&11&13\\
   		\hline % 在表格最下方绘制横线
   		12&24&13&22&14&19&15&23\\
   		\hline % 在表格最下方绘制横线
   		16&8&17&12&18&9&19&7\\
   		\hline % 在表格最下方绘制横线
   		20&18&21&11&22&21&23&29\\
   		\hline % 在表格最下方绘制横线
   		24&3&25&25&26&26&27&15\\
   		\hline % 在表格最下方绘制横线
   		28&14&29&27&30&30&x&x\\
   		\hline % 在表格最下方绘制横线
   	\end{tabular}\\
   	There is a bijection between plaintext m and ciphertext c, so we can decrypt the massage by the table above.
    
    \textbf{Part Four}\\
    $\sqrt{4369}<\sqrt{4883}<70$\\
    Find all prime integars smaller than 70, they are 2, 3, 5, 7, 11, 13, 17, 19, 23, 29, 31, 37, 41, 43, 47, 53, 59, 61, 67.\\
    For 4369, we divide 4369 with each prime integar above. And we find 4369=17*257. Since 2,3,5,7,11,13 are not factor of 257, 257 is prime. 4369=17*257.\\
    For 4883, we divide 4883 with each prime integar above. And we find 4883=19*257. 4883=19*257.
    
    
    
    \textbf{Part Five}\\
    Try p=2,3,5,7,>7\\
    When p=2,\\
    \begin{equation}       
    \left(                
    \begin{array}{ccc}
    3 & 5\\  
    7 & 3\\  
    \end{array}
    \right) 
    \mod
    2
    =
    \left(                
    \begin{array}{ccc}
    1 & 1\\  
    1 & 1\\  
    \end{array}
    \right) 
    \mod
    2
    \end{equation}
    \begin{equation}
    det(\left(                
    \begin{array}{ccc}
    1 & 1\\  
    1 & 1\\  
    \end{array}
    \right) 
    )=0\\
    \end{equation}
    when p=2, it is not invertible\\
    
    When p=3,\\
    \begin{equation}       
    \left(                
    \begin{array}{ccc}
    3 & 5\\  
    7 & 3\\  
    \end{array}
    \right) 
    \mod
    3
    =
    \left(                
    \begin{array}{ccc}
    0 & 2\\  
    1 & 0\\  
    \end{array}
    \right) 
    \mod
    3
    \end{equation}
    \begin{equation}
    det(\left(                
    \begin{array}{ccc}
    0 & 2\\  
    1 & 0\\  
    \end{array}
    \right) 
    )=-2\\
    \end{equation}
    when p=3, it is invertible\\
    
    When p=5,\\
    \begin{equation}       
    \left(                
    \begin{array}{ccc}
    3 & 5\\  
    7 & 3\\  
    \end{array}
    \right) 
    \mod
    5
    =
    \left(                
    \begin{array}{ccc}
    3 & 0\\  
    2 & 3\\  
    \end{array}
    \right) 
    \mod
    5
    \end{equation}
    \begin{equation}
    det(\left(                
    \begin{array}{ccc}
    3 & 0\\  
    2 & 3\\  
    \end{array}
    \right) 
    )=9\\
    \end{equation}
    when p=5, it is invertible\\
    
    When p=7,\\
    \begin{equation}       
    \left(                
    \begin{array}{ccc}
    3 & 5\\  
    7 & 3\\  
    \end{array}
    \right) 
    \mod
    7
    =
    \left(                
    \begin{array}{ccc}
    3 & 5\\  
    0 & 3\\  
    \end{array}
    \right) 
    \mod
    7
    \end{equation}
    \begin{equation}
    det(\left(                
    \begin{array}{ccc}
    3 & 5\\  
    0 & 3\\  
    \end{array}
    \right) 
    )=9\\
    \end{equation}
    when p=7, it is invertible\\
    
    when $p > 7 (p \in Z)$,\\
    \begin{equation}       
    \left(                
    \begin{array}{ccc}
    3 & 5\\  
    7 & 3\\  
    \end{array}
    \right) 
    \mod
    p
    =
    \left(                
    \begin{array}{ccc}
    3 & 5\\  
    7 & 3\\  
    \end{array}
    \right) 
    \mod
    p
    \end{equation}
    \begin{equation}
    det(\left(                
    \begin{array}{ccc}
    3 & 5\\  
    7 & 3\\  
    \end{array}
    \right) 
    )=-26\\
    \end{equation}
    when $p > 7 (p \in Z)$, it is invertible.\\
    
    Overall, it is not invertible only when p=2. Otherwise, it is invertible.
    
    
    \textbf{Part Six}\\
    Show that either a or b is congruent to 0 mod p means that we only need to prove either $ p|a $ or $p|b$.\\
   	From assignment 1 ex1.3, we show that if a, b, n are three integars, $n | ab$, gcd(a, n) = 1, then $n | b$. \\
    Let n=p, $p|ab$, since p is prime, gcd(p,a)=1 or gcd(p,a)=p. If gcd(p,a)=1, then $p | b$ according to above thereom, and $b\equiv0\mod p$. If gcd(p,a)=p, then $p|a$, and $a\equiv0\mod p $.
    
	\textbf{Part Seven}\\
	$2^{2017}\equiv 2^{2017}\mod 5 \equiv 2*4^{1008}\mod 5 \equiv 2*(-1)^{1008}\mod 5\equiv 2\mod 5$ \\
	$2^{2017}\equiv 2^{2017}\mod 13 \equiv 2*64^{336}\mod 13 \equiv 2*(-1)^{336}\mod 13\equiv 2\mod 13$ \\
	$2^{2017}\equiv 2^{2017}\mod 31 \equiv 4*32^{403}\mod 13 \equiv 4*(1)^{334}\mod 31\equiv 4\mod 31$ \\
    Since 5*13*31=2015, we apply chinese remainder thereom to solve $2^{2017}\mod 2015 $\\
    M=2015\\ 
    $M_1 =\frac{M}{m_1}=\frac{2015}{5}=403$\\
    $M_2 =\frac{M}{m_2}=\frac{2015}{13}=155$\\
    $M_3 =\frac{M}{m_3}=\frac{2015}{31}=65$\\
    $403*t_1\equiv 1 \mod 5$ t1=2+5*n$\forall n \in Z$\\
    $155*t_2\equiv 1 \mod 13$ t2=-1+13*n$\forall n \in Z$\\
    $65*t_3\equiv 1 \mod 31$ t3=-10+31*n$\forall n \in Z$\\
    (Solve $t_1,t_2,t_3$ by extended euclidean algorithm.)\\
    We calculate x=$(\sum_{1}^{3}a_i t_i M_i)+M*k(k\in Z)$=2*2*403+2*(-1)*155+4*(-10)*65+2015*k=-1298+2015*k=717+2015*k\\
    $2^{2017}\equiv 717 \mod 2015$\\
    $2^{2017}\mod 2015$=717
    
\end{homeworkProblem}

\pagebreak


\begin{homeworkProblem}
1.\\
Rabin cryptosystem is an asymmetric cryptosystem. It contains public key and private key: public key is used to encrypt the plaintext and it is known by everybody. While the private key is used to decrypt the ciphertext and it is only known by text recipient and owner of plaintext.\\
Choose two random large prime numbers p, q as the private keys, and n=p*q is the public key.
For encryption, if the plaintext $m\in [0,n-1]$, the ciphertext c could be generated by\\
$c=m^{2}\mod n$ \\
Proposition: \\
Suppose p is a prime that $p\equiv 3 \mod 4$. If y has a square root (r) mod p ($y\equiv r^2 \mod p$), let $x\equiv y^{(p+1)/4}\mod p$, then $r\equiv +-x \mod p$.\\
 If y has not a square root (r) mod p ($-y\equiv r^2 \mod p$), let $x\equiv y^{(p+1)/4}\mod p$, then $r\equiv +-x \mod p$\\
 For decryption, the private key is necessary to help us decrypt efficiently by applying proposition above.\\
 $c\equiv m^2\mod n$, we can find corresponding plaintext m efficiently only if we know p and q.\\
 Since n=p*q, we know that
 $c\equiv m^2\mod p$\\
 $c\equiv m^2\mod q$\\
 Because of the process of encryption, c has a square root mod n, and c has a square root mod p or q.\\
By applying the proposition, we can get:\\
$m\equiv \pm c^{(p+1)/4}\mod p$ and $m\equiv \pm c^{(q+1)/4}\mod q$\\
Know we can apply Chinese remainder thereom to solve four simultaneous equations.\\
$m\equiv c^{(p+1)/4}\mod p$ and $m\equiv c^{(q+1)/4}\mod q$\\
$m\equiv c^{(p+1)/4}\mod p$ and $m\equiv -c^{(q+1)/4}\mod q$\\
$m\equiv -c^{(p+1)/4}\mod p$ and $m\equiv c^{(q+1)/4}\mod q$\\
$m\equiv -c^{(p+1)/4}\mod p$ and $m\equiv -c^{(q+1)/4}\mod q$\\
$M=n=p*q, M1=\frac{M}{p},M2=\frac{M}{q}$, and use extended euclidean thereom to find t1,t2 such that $M1*t1\equiv 1\mod m1$ and $M2*t2\equiv 1\mod m2$, and finally we can get four $m\equiv M1*a1*t1+M2*a2*t2\mod n$($m\equiv a\mod M$). However, only one of them is correct(meaningful) plaintext.\\
2.\\
a. Through the decryption method above, it would generate four answers, and only one of them is correct(meaningful) answer. Meaningful massage can be expected fairly soon since there is at least 25 percent chance to get the correct answer if we choose one at random.\\
b. No. If Eve only has ciphertext x and public key n without private keys p and q, she needs to solve $x\equiv m^2 \mod n$(m is the plaintext). There is no known method to solve this without private keys. Although she can factorize n to any two prime numbers and try to use method above, since p and q are all big prime numbers, the factorization problem is also hard. So it is hard for Eve to determine the original message.\\
c.Eve can try to use CCA to break the system.\\
If Eve use a ciphertext c to compute the output of the system (possible x) for many times, she would get all four possible outputs $\pm r1, \pm r2$. Randomly choose  $\pm r1$ to minus $\pm r2$\\
If we pick +r1 and +r2 or -r1 and -r2, then $|r1-r2|=q$, if we pick +r1 and -r2 or -r1 and +r2, then $|r1-r2|=p$. So we can calculate $gcd(|r1-r2|,n)$ to get one non-trivial factor, and get another non-trivial factor. Then we finish factorization.

  


\end{homeworkProblem}
\pagebreak
\begin{homeworkProblem}
Let n be numbers of people in the group, we know that n satisfies three equations below at the same time:\\
$n\equiv 1\mod 3$\\
$n\equiv 2\mod 4$\\
$n\equiv 3\mod 5$\\
we can solve n by applying Chinese Remainder thereom:\\
M=3*4*5=60\\
$M_1 =\frac{M}{m_1}=\frac{60}{3}=20$\\
$M_2 =\frac{M}{m_2}=\frac{60}{4}=15$\\
$M_3 =\frac{M}{m_3}=\frac{60}{5}=12$\\
$20*t_1\equiv 1 \mod 3$ t1=-1+3*n$\forall n \in Z$\\
$15*t_2\equiv 1 \mod 4$ t2=-1+4*n$\forall n \in Z$\\
$12*t_3\equiv 1 \mod 5$ t3=3+5*n$\forall n \in Z$\\
(Solve $t_1,t_2,t_3$ by extended euclidean algorithm.)\\
We calculate n=$(\sum_{1}^{3}a_i t_i M_i)+M*k(k\in Z)$=1*(-1)*20+2*(-1)*15+3*(3)*12+60*k=58+60*k\\
$n\equiv 58 \mod 60$\\
We choose n as two smallest positive integar: 58,1nn 18.\\
Two smallest possible numbers of people in group are 58 and 118. 
\end{homeworkProblem}

\end{document}
